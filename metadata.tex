%%% Please fill in basic information on your thesis, which will be automatically
%%% inserted at the right places.

% Type of your thesis:
%	"bc" for Bachelor's
%	"mgr" for Master's
%	"phd" for PhD
%	"rig" for rigorosum
\def\ThesisType{bc}

% Language of your study programme:
%	"cs" for Czech
%	"en" for English
\def\StudyLanguage{cs}

% Thesis title in English (exactly as in the official assignment)
% (Note: \xxx is a "ToDo label" which makes the unfilled visible. Remove it.)
\def\ThesisTitle{Data Structures for Sketching Dynamic Sets}

% Author of the thesis (you)
\def\ThesisAuthor{Vojtěch Gaďurek}

% Year when the thesis is submitted
\def\YearSubmitted{2024}

% Name of the department or institute, where the work was officially assigned
% (according to the Organizational Structure of MFF UK in English,
% see https://www.mff.cuni.cz/en/faculty/organizational-structure,
% or a full name of a department outside MFF)
\def\Department{Computer Science Institute of Charles University}

% Is it a department (katedra), or an institute (ústav)?
\def\DeptType{Institute}

% Thesis supervisor: name, surname and titles
\def\Supervisor{Pavel, Veselý, Mgr. Ph.D.}

% Supervisor's department (again according to Organizational structure of MFF)
\def\SupervisorsDepartment{Computer Science Institute of Charles University}

% Study programme (does not apply to rigorosum theses)
\def\StudyProgramme{Computer Science}

% An optional dedication: you can thank whomever you wish (your supervisor,
% consultant, who provided you with tea and pizza, etc.)
\def\Dedication{%
I want to thank all who were helpful namely Mgr. Jiří Beneš, Mgr. Radek Zikmund, especially my supervisor, Mgr. Pavel Veselý, Ph.D. for all he has done and all the consultations he provided. I am very glad to have him behind my back.

We used AI tools quite extensively in this project. We primarily used them to refractor, transform, or rewrite already-written text or code. We have not primarily used AI for idea generation. The tools we used were ChatGPT, Grammarly, and GitHub Copilot. This list is not enumerative and specifies only the most used.
}

% Abstract (recommended length around 80-200 words; this is not a copy of your thesis assignment!)
\def\Abstract{%
Our work explores algorithms for finding the symmetric differences between two large sets that differ only in a few elements. We focus on performance and modularity from an implementation perspective and memory complexity from an algorithmic perspective. We achieved fast and modular code using advanced C\# features, mainly expression trees, allowing us to achieve update speeds of $152$ MOp/s $k$. We compared different hash functions, finding tabulation to have the best properties for recovery.  Finally, the algorithms were applied to structured data, improving the ratio of the memory size to the number of elements in the difference from $1$:$0.8$ to $1$:$1.5$ for $k$-mers.}

% 3 to 5 keywords (recommended) separated by \sep
% Keywords are useful for indexing and searching for the theses by topic.
\def\ThesisKeywords{%
Set Recovery, Expression trees, Bloom Filters, k-mers 
}

% If any of your metadata strings contains TeX macros, you need to provide
% a plain-text version for use in XMP metadata embedded in the output PDF file.
% If you are not sure, check the generated thesis.xmpdata file.
\def\ThesisAuthorXMP{\ThesisAuthor}
\def\ThesisTitleXMP{\ThesisTitle}
\def\ThesisKeywordsXMP{\ThesisKeywords}
\def\AbstractXMP{\Abstract}

% If your abstracts are long and do not fit in the infopage, you can make the
% fonts a bit smaller by this setting. (Also, you should try to compress your abstract more.)
\def\InfoPageFont{}
%\def\InfoPageFont{\small}  % uncomment to decrease font size

% If you are studing in a Czech programme, you also need to provide metadata in Czech:
% (in English programmes, this is not used anywhere)

\def\ThesisTitleCS{Datové struktury pro sketching dynamických množin}
\def\DepartmentCS{Informatický ústav Univerzity Karlovy}
\def\DeptTypeCS{Ústav}
\def\SupervisorsDepartmentCS{Informatický ústav Univerzity Karlovy}
\def\StudyProgrammeCS{Informatika}

\def\ThesisKeywordsCS{%
Rekontrukce množiny, Expression trees, Bloomovy Filtry, k-mery 
}

\def\AbstractCS{
Naše práce se zabývá algoritmy pro nalezení symetrických rozdílů mezi dvěma velkými množinami, které se liší pouze v několika prvcích. Zaměřujeme se na výkon a modularitu z hlediska implementace a paměťovou složitost z hlediska algoritmů. Rychlého a modulárního kódu jsme dosáhli pomocí pokročilých funkcí C\#, zejména Expression trees, což nám umožnilo dosáhnout rychlosti aktualizace $152$ MOp/s $k$. Porovnali jsme různé hashovací funkce. Nakonec byly algoritmy aplikovány na strukturovaná data, což zlepšilo poměr velikosti paměti k počtu prvků v rozdílu z $1$:$0.8$ na $1$:$1.5$ pro $k$-mery.
}
