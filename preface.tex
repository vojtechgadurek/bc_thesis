\chapter*{Introduction}
\addcontentsline{toc}{chapter}{Introduction}

\section*{Motivation}

Advances in technology in the last decade led to a huge accumulation of data. 
This has created the need for algorithms that can work with large data in small memory. 
In our work, we focus on the problem of finding the symmetric difference of two large sets 
that differ only in a small number of elements. 
Finding the symmetric difference between two sets has many applications 
from keeping instances of distributed databases in sync to comparing closely related genomes.

\section*{Basic Aproach}
The state-of-the-art solution to such a problem is 
to use an Invertible Bloom Lookup Table (IBLT).\ref{iblt} This data structure is very close to a hash table.
It differs mainly in the way how it deals with collisions. It just adds the values of the colliding keys and keeps the count of keys in each bucket.
This means that when an item is added to the table and then removed, the table will be the same as the table without these two operations.
This property is very useful in our case because we can add all elements from the first set to the table and then remove all elements from the second set; only the items in the difference remain in the table. We then may try to recover the difference from the table, returning all values in buckets where only one key was hashed. This, however, leads to some keys being unrecovered if multiple keys were hashed to the same bucket.
Another crucial difference in the table is that we do not use one hash function but $k$ hash functions, for example, $k=3$. So, each key is stored in $k$ buckets, 
improving the probability no two keys are hashed into the same buckets.
We may use such a structure to find the symmetric difference of two sets in space linear in the size of the difference.
The authors Houen, Pagh, and Waltzer built on the IBLT and proposed a new data structure that improves the space consumed by a constant. \cite{doi:10.1137/1.9781611977585.ch21} \ref{hwp}
Their improvement lies in replacing the summation of values in the buckets with the XOR operation and not keeping the count of keys in the buckets.
This may lead to the recovery of keys, not in the difference, but the authors successfully showed such an event does not happen with high probability.


\section*{Implemenation}
We expected a typical use case to be run on large data, therefore we highly prioritized performance. By combining low-level optimization and parallelism, we achieved a speed of 130 million encoded $k$-mers per second on $4$ cores with $32GB$ of RAM.  See: \ref{encoding_kmer_sets}

However, we also made the library highly modular, allowing users to replace each part easily. Our main and only programming language was C\#. We used its runtime code generation features extensively, namely Expression Trees. 

We use them in our custom-made library, FlashHash, which provides access to many hash functions from two hash function families. These hash functions are compiled during runtime for better performance. \ref{FlashHash}

Working with Expression Trees may be tedious and cumbersome. As such, we developed a small library—LittleSharp—to make working with them more enjoyable and safe. It provides a reasonable type of safety, making many otherwise runtime errors a compile-time error. \ref{LittleSharp}

Lastly, we implemented a library for reading superstrings of $k$-mers called RedaFasta.  \ref{RedaFasta}

\section*{Improvements}

\subsection*{Recovery with oracle}
In Bioinformatics, one often work with $k$-mer sets instead of DNA strings. $k$-mer is DNA substring of lenght $k$. Finding symmetric difference of two $k$-mer sets is useful for useful for closness of different species. 
We were able to use the fact that data in such sets are not entirely random, and from the existence of some $k$-mer, we may infer the existence of other $k$-mers. We called such a process a recovery with 
an oracle. With this, we beat the theoretical bound of $1$:$1$ ratio by the ratio $1$:$1.52$ for $31$-mer sets. We found this value experimentally.
 \ref{fig:oracle_recovery}


\section*{Other results}
We found no difference between IBLT and HPW regarding the probability of recovery. However, encoding to IBLT is much more expensive. 

We found that different hash function families significantly influence recovery and performance. The best hash function family for recovery with Oracle was tabulation,  which was, however, more than 10 times slower than the fastest hash function family multiply-shift. They, however, had comparable recovery for nor